\documentclass[10pt,a4paper]{article}
\usepackage[utf8]{inputenc}
\usepackage[spanish]{babel}
\usepackage{amsmath}
\usepackage{amsfonts}
\usepackage{amssymb}
\usepackage{enumitem}
\usepackage{hyperref} 
\author{Pablo Riutort Grande}
\title{Prueba de Evaluación Contínua 3\\ \vspace{1cm}\textbf{DERECHO PENAL E INTERNET}}
\begin{document}
\maketitle
\pagebreak
\section{Cuestiones}
\begin{enumerate}

%1
\item \textbf{El término “ciberdelito” o “cibercrimen”:}\\
\textit{a) Designa conductas ilícitas vinculadas a las TICs, recogidas en el Código Penal.}\\
Las conductas ilícitas que corresponden a algunos artículos del código penal se denominan ciberdelitos. Estos artículos son aplicables al contexto de las TIC y otros directamente recogen el caso concreto del uso de la informática para cometer un hecho ilícito (\cite[Título XIII. Cap. VI. Art. 248.2.a]{cp} por ejemplo).

%2
\item \textbf{Explica en qué consiste el delito conocido popularmente como “sexting”. Debes basarte y hacer referencia al artículo del CP que recoge este delito.}\\
Este delito queda recogido en el artículo 197.7 del código penal y consiste en difundir imágenes o grabaciones a terceros de una persona sin su consentimiento cuyo contenido pudiera menoscabar gravemente su intimidad. Cabe recalcar que estas imágenes o grabaciones pueden haberse obtenido con el consentimiento de la víctima \cite[Código Penal. Título X. Cap I. Art. 197.7]{cp}.

%3
\item \textbf{Explica el concepto de “pornografía técnica” descrito en la Circular de la Fiscalía 2/2015
relativa a la pornografía infantil.}\\
La pornografía técnica se refiere a todo aquel material visual que presente a una persona que aparente ser menor de edad participando en una conducta sexual explícita o cualquier representación de los órganos sexuales de la misma siempre y cuando sean con fines sexuales \cite[Código Penal. Título VIII. Cap V. Art. 189.1.c)]{cp}.\\
Dependerá de cada estado miebro de la unión decidir si se procede a sancionar cuando la persona aparentemente menor resulte ser mayor de edad en el momento en que se realizó el material \cite[Art 5.7 de la Directiva 2011/93/UE]{dir}.

%4
\item \textbf{¿En qué consiste la conducta del delito de intrusismo informático? De acuerdo con el artículo que lo recoge ¿cuáles son los elementos requeridos en el mismo para entender que se ha cometido dicho delito?}\\
Las conductas del delito de instrusismo informático consisten en conductas de acceso no autorizado a sistemas informáticos sin motivo ulterior más allá que el del aprendizaje del funcionamiento de sistemas informáticos y sus componenetes.\\

El artículo del código penal que lo recoge es el 197.3 \cite{cp}. Según este, no está muy clara cuál es la interpretación de los elementos requeridos para entender que se ha cometido un delito. Debido a encontrarse este artículo en un capítulo destinado a los delitos contra la intimidad, técnicamente si el autor no vulnera este derecho no hay un acto ilícito. Además, el artículo 197.3 penaliza el acceso a datos o programas informáticos contenidos en un sistema informático lo cual es un tanto ambíguo dadas las circunstancias siendo esta interpretación muy amplia.

%5
\item \textbf{Si como administrador de sistemas tienes conocimiento de que una persona aloja contenidos pornográficos de adultos en su página web. ¿Deberás ponerlo en conocimiento de las fuerzas y cuerpos de seguridad?}\\
El contenido pornográfico para adultos no es un contenido ilícito, por tanto, no incurre en una responsabilidad penal. No corresponde a las fuerzas y cuerpos de seguridad el gestionar y/o retirar este contenido, corresponde a los usuario el filtrarlo debidamente. Por tanto, no será necesario ponerlo en conocimiento.

%6
\item \textbf{Un empresario sospecha que un trabajador está utilizando el correo electrónico de la empresa para enviar información confidencial de la empresa a la competencia. Para comprobarlo decide acceder al buzón de correo del trabajador.}\\
\textit{c) La conducta únicamente podría estar justificada en aquellos casos en que
exista en la empresa una política clara sobre el uso de las TIC, se haya
informado de la misma a los trabajadores y, además, se haya advertido
explícitamente sobre la posibilidad de control del correo asignado a los
trabajadores por parte del empresario.}\\
En este caso entran en conflicto el artículo 20 de la Ley del estatuto de los trabajadores y los artículos 197 a 201 del código penal. En el primero, el empresario tiene derecho a adoptar las medidas de vigilancia y control que considere adecudas para evaluar el cumplimiento del trabajo de su empleado \cite{trabajo} pero, a su vez, el empleado tiene derecho a la intimidad en su puesto de trabajo y a que no se acceda a su bandeja de correo se tengan o no sospechas de espionaje industrial.\\
La mejor solución consiste en mantener una política clara de empresa en estos casos y habilitar al empresario de mecanismos que permitan ejercer sus derechos sin vulnerar los del trabajador. Siendo el acceso a la bandeja de entrada algo demasiado intrusivo e ilícito en este caso de sospecha, quizá podría optarse por instaurarse algún tipo de filtrado de correos u otra técnica menos lesiva a la intimidad.

%7
\item \textbf{ Laura es estudiante de un máster en aplicaciones móviles. Ha creado un programa para dispositivos móviles que es capaz de cifrar todos los archivos del dispositivo, y que puede llegar a inutilizar el mismo. La conducta de Laura es: }\\
\textit{c) Podría ser punible siempre y cuando se demuestre que la intención principal
de Laura al crear el programa era facilitar la comisión de alguno de los
delitos de daños informáticos.}\\
De acuerdo con el artículo 264 del código penal, encontrándose bajo el título ``Delitos contra el patrimonio y el orde socioecnómico'', si el programa en cuestión ha sido concebido con la intención de cometer alguno de los delitos tipificados en el mismo artículo, es motivo de castigo \cite[Art. 264 ter.b]{cp}. Por tanto, con que se demuestra que había una intención de hacer daño es suficiente para condenar a Laura.

\newpage

%8
\item \textbf{La actividad de un motor de búsqueda, consistente en rastrear automáticamente la red para indexar, sin previa selección de los contenidos, todos los contenidos disponibles con el objetivo de ofrecerlos debidamente ordenados y sistematizados a los internautas, constituye una actividad:}\\
\textit{b) Podría llegar a ser punible si el prestador del servicio llegara a tener
conocimiento efectivo de la ilicitud del contenido al que redirecciona y no
suprimiese o inutilizara el enlace a los contenidos ilícitos.}\\
El \textit{Plan de acción para promover una utilización segura de Internet de la Comunicación de la Comisión al Parlamento Europeo y al Consejo} establece que el contenido ilegal debe ser tratado en su origen por las autoridades de policía y las judiciales, es decir, por el Estado. Por tanto, los prestadores de los servicios del motor de búsqueda, quedarán vinculados a Ley 34/2002, de 11 de julio, de servicios de la sociedad de la información y de comercio electrónico, donde se expone que los prestadores de los servicios no tendrán responsabilidad por la información a la que redirigen estos enlaces exceptuando si se tiene conocimiento efectivo de que esa información sea ilícita \cite[Cap. II. Sección 2. Art. 17]{society}.

\end{enumerate}
\section{Supuesto práctico}
\begin{enumerate}
\item \textbf{¿De qué delito podría ser constitutiva la conducta de Francisco José? ¿Por qué?}\\
Francisco José, un trabajador con la tarea de arreglar el ordenador de Matilde, encuentra unos archivos de marcado contenido sexual y los copia sin consentimiento de su cliente. Estos archivos son compartidos entre los contactos de Francisco y estos, a su vez, hacen lo mismo hasta llegar al punto en que todo el pueblo donde residía Matilde llega a ver el contenido.\\

Debido a que los hechos tuvieron lugar en julio de 2018 se aplicarán los artículos y definiciones recogidos en el Reglamento
general de protección de datos, aplicable desde el 25 de mayo de 2018.\\
En este reglamento se definen los siguientes conceptos \cite[Art. 4]{dirregp}
\begin{itemize}
\item Datos personales: toda información sobre una persona física identificada o identificable.
\item Persona física: es una persona física identificable toda persona cuya identidad pueda determinarse, directa o indirectamente.
\item Tratamiento (de datos): Cualquier operación o conjunto de operaciones realizadas sobre datos personales o conjuntos de
datos personales entre los que se incluye la comunicación por transmisión, difusión u otro forma de habilitación de acceso.
\item Consentimiento: Es toda manifestación de voluntad libre, específica, informada e inequívoca por la que el interesado acepta, ya sea mediante una declaración o una clara acción afirmativa, el tratamiento de datos personales que le conciernen
\item Terceros: Persona física distinta del interesado y del responsable
del tratamiento.
\end{itemize}
Entendemos, por tanto, que los datos personales son las fotos de Matilde y que además estas pueden identificarla a ella directamente, aplicándole también la definición de persona física. También hubo un tratamiento (difusión) sin el consentimiento del interesado, en este caso, Matilde.\\
Dada la naturaleza de estos datos, se puede aplicar el artículo 9.1 del mismo reglamento, donde se expone textualmente que queda prohibido todo tratamiento de datos relativos a la vida sexual \cite[Art. 9.1]{dirregp}.\\

Francisco José, al compartir los archivos sensibles de Matilde a terceros, efectuó una transmisión no autorizada y, por tanto, un tratamiento ilícito \cite[Art 6.]{dirregp}. Esta difusión podría considerarse una falta a la intimidad y a la integridad moral de su cliente, por lo que se podría aplicar el artículo 197.7 del código penal que castiga a aquel que difunda sin autorización de la persona afectada imágenes que menoscaben gravemente la intimidad de esta persona \cite[Título X, Cap. I. Art. 197.7]{cp} y, por ende, el artículo 173.1 sobre las torturas y otros delitos contra la integridad moral del mismo código donde se expone que quien infringiera a otra persona un trato que menoscabe su integridad moral también tendrá pena de prisión \cite[Título VII. Art. 173]{cp}.\\

Además, el artículo 199 del código penal que expone que el que revele secretos ajenos, de los que tenga conocimiento por su oficio será penado y que aquel profesional que, con obligación de sigilo o reserva, divulgue los secretos de otra persona será también castigado \cite[Título X. Cap. I. Art. 199]{cp}. Dado que Francisco  estaba efectuando una actividad profesional en el momento en que cogió los datos de la papelera de Matilde, y además esos datos claramente son de una naturaleza sensible (independientemente de la profesión que se tenga) y secreta, por su localización y contenido, se podría aplicar este artículo.

\item \textbf{Respecto a las personas que han compartido las imágenes, ¿podrían haber infringido algún artículo del CP? ¿Por qué?}\\
Aunque haya sido Francisco José el primero en difundir los datos sensibles de Matilde eso no exime a aquellas personas que sigan difundiendo esas fotos entre sus contactos de que se les aplique los mismos artículos 197.7 y 173 anteriormente comentados. Todas las difusiones (la primera y las siguientes) son sin la autorización de la víctima y todas son igual de denigrantes hacia su persona y menoscaban su integridad moral. De hecho, se podría interpretar cada difusión como una instancia del mismo suceso entre Matilde y los difusores aplicando el artículo 28.b del código penal de las personas criminalmente responsables de los delitos que dice que \textit{son autores los que cooperan a su ejecución con un acto sin el cual no se habría efectuado} \cite[Título II. Art. 27. ]{cp}.\\
\end{enumerate}

\newpage

\begin{thebibliography}{9}
\bibitem{cp}
  \textit{Código Penal.}\\
  \href{https://www.boe.es/buscar/pdf/1995/BOE-A-1995-25444-consolidado.pdf}{Ley orgánica 10/1995 de 23 de noviembre del Código Penal. Texto consolidado.}

\bibitem{dir}
\textit{Decisión marco 2004/68/JAI del Consejo.}\\
\href{https://www.boe.es/doue/2011/335/L00001-00014.pdf}{DIRECTIVA 2011/92/UE DEL PARLAMENTO EUROPEO Y DEL CONSEJO}

\bibitem{dirregp}
\textit{Reglamento general de protección de datos.}\\
\href{https://www.boe.es/doue/2016/119/L00001-00088.pdf}{REGLAMENTO (UE) 2016/679 DEL PARLAMENTO EUROPEO Y DEL CONSEJO de 27 de abril de 2016}

\bibitem{trabajo}
\href{https://www.boe.es/eli/es/rdlg/2015/10/23/2/con}{Real Decreto Legislativo 2/2015, de 23 de octubre, por el que se aprueba el texto refundido de la Ley del Estatuto de los Trabajadores.}

\bibitem{society}
\textit{Ley 34/2002, de 11 de julio, de servicios de la sociedad de la información y de comercio
electrónico. Jefatura del Estado.}\\
\href{https://www.boe.es/eli/es/l/2002/07/11/34/con}{BOE-A-2002-13758}

\end{thebibliography}

\end{document}